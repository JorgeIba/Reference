{\normalsize
    $ gcd(a,b) = gcd(a, b-qa)$
    \hspace{4em} Bars and Stars: $ \binom{n+k-1}{n} $
    
    Recta: $ax + by = c$ y dos puntos, entonces $a = y_2-y_1$, $b = x_1-x_2$ y $c = x_1y_2-x_2y_1$
    
    Si $S_{n}^{k} = \sum_{i=1}^{n} i^k$ entonces: $S_{n}^{k} = \frac{ (n+1)^{k+1} - 1 - \sum_{j=0}^{k-1} {{k+1} \choose j} S_n^j }{ k+1 }$

    Lucas Theorem: A binomial coefficient ${\tbinom {m}{n}}$ is divisible by a prime p if and only if at least one of the base p digits of n is greater than the corresponding digit of m.
    In particular, ${\displaystyle {\tbinom {m}{n}}}$ is odd if and only if the binary digits (bits) in the binary expansion of n are a subset of the bits of m.

    Wilson Theroem: Un número es primo si y sólo si $(n-1)! \equiv -1 \; (mod \; n)$

    ${\tbinom {n}{m}} = {\tbinom {n_0}{m_0}} {\tbinom {\lfloor \frac{n}{p} \rfloor }{\lfloor \frac{m}{p} \rfloor}}$ $n_0$ is $n$ modulo p and $m_0$ is $m$ modulo p.

    $\displaystyle\prod_{n=0}^{\infty} (1-x^{2kn+k-l})(1-x^{2kn+k+l})(1-x^{2kn+2k}) = \displaystyle\sum_{n=-\infty}^{\infty}(-1)^nx^{kn^2+ln}$

    Las sumas de todos los coprimos con n: $\displaystyle h(n) = \sum_{gcd(n,q)=1} q = \frac{1}{2} * (n  * \varphi(n) + [n==1])$

    Hassel Diagrams - Posets.

    $\dbinom n 0 + \dbinom n 2 + \dbinom n 4 + \dotsb = 2^{n - 1}$

    $\displaystyle f(n) = \sum_{i=1}^{n} |\mu(i)| = \sum_{i=1}^{  \lfloor \sqrt{n} \rfloor} \mu(i) \lfloor \frac{n}{i^2} \rfloor$

    Si $a$ es real y alguno de $n$ o $m$ es un entero positivo $\displaystyle \lfloor \frac{\lfloor \frac{a}{m} \rfloor }{n} \rfloor = \lfloor \frac{a}{mn} \rfloor$

    $ \operatorname{lcm}(a,b,c,d)=\frac{a\cdot b\cdot c\cdot d\cdot\operatorname{gcd}(a,b,c)\cdot\operatorname{gcd}(a,b,d)\cdot\operatorname{gcd}(a,c,d)\cdot\operatorname{gcd}(b,c,d)}{\operatorname{gcd}(a,b)\cdot\operatorname{gcd}(a,c)\cdot\operatorname{gcd}(a,d)\cdot\operatorname{gcd}(b,c)\cdot\operatorname{gcd}(b,d)\cdot\operatorname{gcd}(c,d)\cdot\operatorname{gcd}(a,b,c,d)} $

    Hensel's Lemma: Si tenemos una solución para $f(a) \equiv 0 \pmod{p^k}$ y se cumple que $f'(a) \not\equiv 0 \pmod{p^{2k}}$ entonces se tiene que $a = r \pmod{p^k}$, donde $r$ es la solución a $f(r) \equiv 0 \pmod{p^{2k}}$.

    Propiedades de números de Fibonacci (0-indexed): 

    \begin{center}
        $F_{n-1}F_{n+1} - F_n^2 = (-1)^n$

        $F_{n+k} = F_kF_{n+1} + F_{k-1}F_n$

        $gcd(F_m, F_n) = F_{gcd(m,n)}$
    \end{center}


    Para usar 2SAT recuerda que:

    \begin{center}
        $x \lor y     \longrightarrow (\lnot x \implies y) \land (\lnot y \implies x)$

        $x = 1        \longrightarrow (x \lor x)$

        $x \oplus y   \longrightarrow (x \lor y) \land (\lnot x \lor \lnot y)$

        $x = y        \longrightarrow (x \lor \lnot y) \land (\lnot x \lor y)$

        A lo más uno:
        $(\lnot x \lor \lnot y)$


        DeMorgan's Law:
        $\lnot (x \lor y)  = \lnot x \land \lnot y$

        $\lnot (x \land y) = \lnot x \lor \lnot y$
    \end{center}
    
    Hall's Theorem: Si tenemos un grafo bipartito con un lado X y del otro lado Y, supongamos que |X| <= |Y|. Al tomar un subconjunto S de X, denotamos a N(S) como los nodos en Y con aristas en S. Si para todo S \subseteq X, tenemos que N(S) >= S, entonces hay un emparejamiento que empareje a todo X.

    Pell Equations ($x^2 - n y^2 = 1$): Remember that $(x^2 - n y^2)(a^2 - n b^2) = (xa + nyb)^2 - n(xb + ay)^2$, si haces esto donde $a$ y $b$ son las soluciones de los pasos "anteriores", tienes una recursión que genera todas las soluciones (en los enteros positivos).

    Theorem Lam - Postnikov: The vertices of an alcoved polytope have only integer coordinates. An alcoved polytope es a polytope obtained by inequalities of the form:

    $\alpha_{ij} \leq x_i + x_{i+1} + x_{i+2} + \cdots + x_j \leq \beta_{ij}$

    Esto nos ayuda al momento de querer aplicar simplex, donde las inequaciones son "continuas" y cada $x_i \in \{0, 1\}$.

    Recuerda que ${\binom {x}{k}}$ es un polinomio $x$ de grado $k$. Lo puedes sacar sabiendo que ${\binom {x}{k}} = \frac{x(x-1)(x-2)...(x-k+1)}{k!}$

    Generalización Teorema de Wilson. Sea $n!_p$ el producto de todos los números $x \leq n$ tal que $gcd(p,x)=1$. Entonces se tiene que $(p^k - 1)!_p \equiv -1 \mod{p^k}$.

}