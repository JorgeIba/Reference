\documentclass[12pt, twoside]{article}
\usepackage[utf8]{inputenc}
\usepackage[english]{babel}
\usepackage{hyperref}
\usepackage{filecontents}
\usepackage{amssymb, amsmath, amsbsy}
\usepackage{amsfonts}
\usepackage{multirow,array}
\usepackage{enumerate, multicol}
\usepackage{amssymb}
\usepackage{lscape}
\usepackage{float}
\usepackage{graphicx}
\usepackage{fancyhdr}
\usepackage{booktabs}
\usepackage{verbatim}
\usepackage{siunitx}
\usepackage{cite}
\usepackage{cancel}
\usepackage{longtable}
\usepackage{algpseudocode}
\usepackage{algorithm}
\usepackage{tikz}
\usepackage{xcolor}
\usepackage{tcolorbox}
\usepackage{colortbl}
\usepackage{caption}
\usepackage{enumitem}
\usepackage{pgfplots}
\pgfplotsset{compat=1.18}


\begin{document}
  \textbf{Suffix Automaton} \\
  \textbf{Number of times a substring appears in a string} We can calculate
  the number of times a string appears in a Suffix Automaton with the next dp:
  \begin{align*}
    dp[u] = \sum_{v}{dp[v]} + isAcState[u]
  \end{align*}
  \textbf{Largest common substring of a set of strings} This can be done by
  concatenating all the strings separated by a special character, distinct for
  each string. Then we can with a dfs and Dynamic programming to see which
  special characters can be reached from a certain state without going through
  special characters.

  \textbf{Total length of all different substrings} For this we will need 2
  dp's, $d[v]$ will store the number of different substrings (count paths) and
  $ans[v]$ the total length of this substrings
  \begin{align*}
    dp[u] = \sum_{v} dp[v] \\
    ans[u] = \sum_{v} d[v] + ans[v]
  \end{align*}

  \textbf{Lexicographical smallest shift} Take $S + S$ and build its
  automaton, then we have to find the lexicographical smallest path of size
  $|S|$

  \textbf{Automaton with preffix function}\\
  It's useful to calculate operations on strings which are created
  recursively. For example if we want to calculate the number of times a
  certain pattern appears in a recursive string we can calculate the automaton
  for the pattern, and then use two DP's. One will store the value of the
  preffix functionafter processing the $i$-th recursive string and being in
  the $j$-th state ($G[i][j]$). Then in another DP $T[i][j]$ we will store how
  many times a the pattern appears in the $i$-th recursive string starting in
  the $j$-th state.
\end{document}

